\pagenumbering{roman}  
\chapter*{Abstract}

\addcontentsline{toc}{chapter}{Abstract}

A seguito della larga informatizzazione dei dati all'interno di tutti i settori, oggi la maggior parte dei nostri dati (sensibili e non) \`e memorizzata in qualche server in rete, del quale a volte non ne conosciamo n\`e ubicazione n\`e proprietario. 
Con lo sviluppo di questa tendenza, per\`o, \`e sempre pi\`u a rischio la sicurezza di tali dati, infatti sono sempre di pi\`u le notizie di attacchi informatici contro server per il recupero dei dati, ove poi viene fatto uso malizioso di essi. Tanto per citarne alcuni recenti sono famosi lo scandalo di dati rubati da Ashley Madison nel luglio 2015 \cite{madison} e il furto alla Sony nel dicembre 2014 \cite{sony}.\\
Per evitare che ci\`o accada si sta sempre pi\`u estendendo l'uso della crittografia, tramite la qual viene garantito all'utente l'uso esclusivo (o con un ristretto gruppo di persone) delle informazioni messe a disposizione dal server. 
Tale metodo per\`o non \`e sufficiente a garantire la confidenzialità di tali dati in quanto nella maggior parte di casi i dati sul server sono in chiaro e se volesse il proprietario del server potrebbe recuperare tali dati e venderli o mostrarli in pubblico. Inoltre tramite questo metodo se un criminale informatico riuscisse a ottenere l'accesso al server, potrebbe tranquillamente recuperare i dati senza che nessuno lo scopra. 
Proprio per questo il metodo migliore \`e crittografare i dati sul server, di modo che anche chi ha accesso a tali dati non li pu\`o recuperare a meno che non possegga la chiave per decifrarli.\\
Di seguito analizzeremo un metodo per realizzare tale sicurezza e anche il modo in cui implementarlo successivamente su di un elaboratore.\\
Il metodo analizzato in questo documento riguarda l'utilizzo della struttura b+tree per l'archiviazione di dati crittografati, in particolare viene utilizzato dal client per ricreare l'indice di un database su una chiave anche non primaria per poi effettuare una ricerca su tale indice.
I dati per\`o, che inizialmente sono salvati su di un file o inseriti manualmente dall'utente, vengono successivamente criptati e spediti al server, di modo che esso non possa recuperare i dati al suo interno. Successivamente a seguito dell'esecuzione di una query da parte del cliente sul database, i dati vengono recuperati a partire dalla radice dell'albero e vengono tenuti in cache per effettuare le query pi\`u velocemente. Quando il client ha terminato di utilizzare il database tutti i nodi dell'albero in memoria vengono scambiati di settore e rispediti al server, crittografandoli anche con un salting diverso. Inoltre per garantire che il server non intuisca quali settori vengono recuperati, vengono effettuate delle "fake searches", cio\`e al momento di recuperare ogni nodo si guardano i figli e se ne ha piu' di uno vengono recuperati alcuni settori che possono non essere utili.
